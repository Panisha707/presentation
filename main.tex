\documentclass{beamer}
\usepackage{listings}
\lstset{
%language=C,
frame=single, 
breaklines=true,
columns=fullflexible
}
\usepackage{subcaption}
\usepackage{url}
\usepackage{tikz}
\usepackage{tkz-euclide} % loads  TikZ and tkz-base
%\usetkzobj{all}
\usetikzlibrary{calc,math}
\usepackage{float}
\newcommand\norm[1]{\left\lVert#1\right\rVert}
\renewcommand{\vec}[1]{\mathbf{#1}}
\usepackage[export]{adjustbox}
\usepackage[utf8]{inputenc}
\usepackage{amsmath}
\usetheme{Boadilla}

\title{Assignment 1}
\author{Pansha Gundelli}
\institute{NIPER Hyderabad}
\date{\today}
\begin{document}


\begin{frame}
\titlepage
\end{frame}
\section{Question}
\begin{frame}
\frametitle{Question}
\begin{block}{cbse/math/10/2006/set2-Q17}
Find the value of $p$ for which the points\\ 
$\vec {A}=\begin{pmatrix}-5\\1\end{pmatrix},
\vec {B}=\begin{pmatrix}1\\p\end{pmatrix},
\vec {C}=\begin{pmatrix}4\\-2\end{pmatrix}$
are collinear.
\end{block}
\end{frame}

\begin{frame}
\frametitle{Solution}
Given:-
$\vec {A}=\begin{pmatrix}-5\\1\end{pmatrix},
\vec {B}=\begin{pmatrix}1\\p\end{pmatrix},
\vec {C}=\begin{pmatrix}4\\-2\end{pmatrix}$
\\Given that the points are collinear, so we create a matrix
\begin{align}
  \vec{M} =  \begin{pmatrix}{\vec{B}-\vec{A}&\vec{C}-\vec{A}} \end{pmatrix}^{\top}
\end{align}
where $rank(\vec{M})=1$.We have the matrix $\vec{M}$ as,
\begin{align}
    \vec{M} = \begin{pmatrix}\myvec{1+5&p-1\\4+5&-2-1}
    \end{pmatrix} \\
     \vec{M} = \begin{pmatrix}\myvec{6&p-1\\9&-3}
     \end{pmatrix}  
\end{align}
\end{frame}

\begin{frame}
\frametitle{Solution}
Now we row reduce the matrix $\vec{M}$,
\begin{align}
\begin{pmatrix}\myvec{6&p-1\\9&-3}
    \end{pmatrix}
    \\
    \overset{R_1\leftrightarrow R_2}{\longleftrightarrow}
\begin{pmatrix}\myvec{9&-3\\6&p-1}\end{pmatrix}
\\
\overset{R_1\rightarrow \frac{R_1}{3}}{\longleftrightarrow}
\begin{pmatrix}\myvec{3&-1\\6&p-1}\end{pmatrix}
\\
\overset{R_2\rightarrow R_2-2R_1}{\longleftrightarrow}
\begin{pmatrix}\myvec{3&-1\\0&p+1}\end{pmatrix}\\
\end{align}
\end{frame}

\begin{frame}
\frametitle{Solution}
\begin{align}
    \overset{R_1\rightarrow \frac{R_1}{3}}{\longleftrightarrow}
\begin{pmatrix}
\myvec{1&\frac{-1}{3}\\0&p+1}
\end{pmatrix}
\end{align}
Since $rank(\
\vec{M})=1$,we have
\begin{align}
 p+1=0 \\
\implies p=-1
 \end{align}\\
 Figure  verifies that the points are indeed collinear for $p$ = $-1$

\end{frame}
\begin{frame}{Graphical solution}   \numberwithin{figure}{section}
\begin{figure}[ht]
    \centering
    \includegraphics[width=\columnwidth]{collinear.png}
    \caption{Collinear}
    \label{Graphical solution}
\end{figure}
\end{frame}



\end{document}